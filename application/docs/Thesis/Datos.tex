%%%%%%%%%%%%%%%%%%%%%%%%%%%%% CARATULA%%%%%%%%%%%%%%%%%%%%%%%%
\textheight 19cm
\pagestyle{empty}
\begin{center}
 {\bf {\fontsize{14}{16.8}\selectfont UNIVERSIDAD NACIONAL DE TRUJILLO}}     
 
    {\bf{\fontsize{14}{16.8}\selectfont Facultad de Ciencias Físicas y Matemáticas}} 

  {\bf{\fontsize{14}{16.8}\selectfont Escuela Profesional de Informática}}
\end{center}  

\begin{figure}[ht]
\begin{center}
\includegraphics[width=.4\textwidth]{unt}
\end{center}
\end{figure}

\vskip 2cm
\begin{center}
  { \bf {\fontsize{17}{20.4}\selectfont{PREDICCIÓN DE LA RESPUESTA CORRECTA DE UNA PREGUNTA DE OPCIÓN MÚLTIPLE MEDIANTE TÉCNICAS DE APRENDIZAJE NO SUPERVISADO ASOCIADO CON LA EXPERIENCIA.}}     
  \vskip 3cm
  %{\hspace{-1.7cm}AUTOR:}\par
  {\bf \fontsize{14}{16.8}\selectfont {\hspace{-2.9cm}AUTOR: José Vicente Clavo Tafur}}} \\
    \vskip 0.2cm
    {\bf \fontsize{14}{16.8}\selectfont {\hspace{-1.7cm} ASESOR: Jorge Luis Gutierrez Gutierrez}}
    
\end{center}   


\vskip 1.1cm
\begin{center}    
{\bf {\fontsize{14}{16.8}\selectfont Trujillo - Perú
\vskip 0.0cm
\hspace*{-0.2cm} 
2019 }}
\end{center} 
\newpage
%%%%%%%%%%%%%%%%%%%%%%%%%%%%%%%%%%%%%%%%%%%%%%%%%%%%%%%%%%%%%%%%%%%%%%%%%%%


%%%%%%%%%%%%%%%%%%%%%%%%%%%%CONTRA CARATULA 1 %%%%%%%%%%%%%%%%%%%%%%%%%%%%%
\newpage
\pagestyle{plain}
\pagenumbering{roman}

\hspace*{6cm}
\vskip 9cm
\begin{center}
   {\bf \doublespacing {\fontsize{17}{20.4}\selectfont{PREDICCIÓN DE LA RESPUESTA CORRECTA DE UNA PREGUNTA DE OPCIÓN MÚLTIPLE MEDIANTE TÉCNICAS DE APRENDIZAJE NO SUPERVISADO ASOCIADO CON LA EXPERIENCIA.}}}     
\end{center} 
\newpage
%%%%%%%%%%%%%%%%%%%%%%%%%%%%%%%%%%%%%%%%%%%%%%%%%%%%%%%%%%%%%%%%%%%%%%%%%%%


%%%%%%%%%%%%%%%%%%%%%%%%%%%%% CONTRA CARATULA 2 %%%%%%%%%%%%%%%%%%%%%%%
\begin{center}
   {\bf {\fontsize{14}{16.8}\selectfont{José Vicente Clavo Tafur}}}\\     
   \end{center}   

\vskip 3.2cm
\begin{center}
   {\bf \doublespacing {\fontsize{17}{20.4}\selectfont{PREDICCIÓN DE LA RESPUESTA CORRECTA DE UNA PREGUNTA DE OPCIÓN MÚLTIPLE MEDIANTE TÉCNICAS DE APRENDIZAJE NO SUPERVISADO ASOCIADO CON LA EXPERIENCIA.}}}     
\end{center}   
  \vskip 2cm
\begin{verse}
 \fontsize{12}{14.4}\selectfont{\hspace*{0.6cm}Tesis presentada a la Escuela Profesional de Informática en la Facultad de Ciencias Físicas y Matemáticas de la Universidad Nacional de Trujillo, como requisito parcial para la obtención del grado de Bachiller en ciencia de la computación ( Título profesional de Ing. Informático)}
\end{verse}

\vskip 1.5cm 
{\fontsize{14}{16.8}\selectfont ASESOR: Jorge Luis Gutierrez Gutierrez} 
 \vskip 1cm 
 \begin{center}    
 \vskip 2cm
{\fontsize{14}{16.8}\selectfont Trujillo - Perú
\vskip 0.2cm
\hspace*{-0.2cm} 
2019}
\end{center} 
\newpage
%%%%%%%%%%%%%%%%%%%%%%%%%%%%%%%%%%%%%%%%%%%%%%%%%%%%%%%%%%%%%%%%%%%%%%%%%%%%%


%%%%%%%%%%%%%%%%%%%%%%%%%%%%HOJA DE APROBACION %%%%%%%%%%%%%%%%%%%%%%%%%%%%%
\begin{center}
 {\bf {\Large HOJA DE APROBACIÓN }     
 \vskip 1.5cm
  {\Large PREDICCIÓN DE LA RESPUESTA CORRECTA DE UNA PREGUNTA DE OPCIÓN MÚLTIPLE MEDIANTE TÉCNICAS DE APRENDIZAJE NO SUPERVISADO ASOCIADO CON LA EXPERIENCIA.}}
 \vskip 1cm 
  {\large{José Vicente Clavo Tafur}}\\

 \vskip 1cm
\end{center} 
Tesis defendida y aprobada por el jurado examinador:
\vskip 1.5 cm
\begin{flushleft} 
$\overline{\mbox{Prof. Dr. XXXXXX - Asesor}}$\\
\vskip -0.5cm
Departamento de Informática - UNT
\end{flushleft} 
\vskip 1cm
\begin{flushleft} 
$\overline{\mbox{Prof. Mg. XXXXXX}}$\\
\vskip -0.5cm
Departamento de Informática - UNT
\end{flushleft} 
\vskip 1cm
\begin{flushleft} 
$\overline{\mbox{Prof. Mg. XXXXXXX}}$\\
\vskip -0.5cm
Departamento de Informática - UNT
\end{flushleft}
\vskip 0.8cm 
\begin{center}    
Trujillo, xX de mayo del 2019
\end{center} 
\newpage
%%%%%%%%%%%%%%%%%%%%%%%%%%%%%%%%%%%%%%%%%%%%%%%%%%%%%%%%%%%%%%%%%%%%%%%%%%%%


%%%%%%%%%%%%%%%%%%%%%%%%%%%% DEDICATORIA %%%%%%%%%%%%%%%%%%%%%%
 
 \addcontentsline{toc}{chapter}{Dedicatoria}
 {\bf\Large {Dedico esta tesis a :}}
 \vskip 1cm
\begin{quotation}
{\it Mis padres quienes en todo momento me apoyaron. En especial a mi madre quien fue mi motor para seguir en mi camino y poder finalizar este proyecto.
\vskip 1cm
Mi hermano con quien siempre compartimos conocimiento para así poder re-solver algunos problemas que aparecieron en el transcurso del desarrollo de este proyecto.
\vskip 1cm
A todas las personas que contribuyen con las ciencia en el Perú y cada día se esfuerzan para que esta siga creciendo }
\end{quotation}
%%%%%%%%%%%%%%%%%%%%%%%%%%%%%%%%%%%%%%%%%%%%%%%%%%%%%%%%%%%%%%%%%%%%%%%%%%%


%%%%%%%%%%%%%%%%%%%%%%%%%%%% AGRADECIMENTOS %%%%%%%%%%%%%%%%%%%%%%
\newpage

 \addcontentsline{toc}{chapter}{Agradecimientos}
 {\bf\Large {\flushleft{Agradecimientos}}}
 \vskip 1.5cm
\begin{quotation}
Agradezco a Dios por haberme bendecido en toda mi vida ....
{\vskip 1cm}
A mis profesores del Departamento de Informática, de los cuales recibí una gran cantidad de conocimientos  . 
\vskip 1cm
A  mi  asesor  Prof.  Dr.  José  Luis  Gutierrez Gutierrez  que  siempre  se  mostró disponible e interesado en ayudarme.
\vskip 1cm
 . . .
 \end{quotation}
%%%%%%%%%%%%%%%%%%%%%%%%%%%%%%%%%%%%%%%%%%%%%%%%%%%%%%%%%%%%%%%%%%%%%%%%%%%


%%%%%%%%%%%%%%%%%%%%%%%%%%%% RESUMEN%%%%%%%%%%%%%%%%%%%%%%
\newpage
\begin{center}
 \addcontentsline{toc}{chapter}{Resumen}
 {\bf\LARGE Resumen}
\end{center} 
\begin{quotation}
\vskip 0.5cm
En la actualidad, hay algunas evaluaciones que son aplicadas en una determinada fecha y los resultados no son entregadas hasta algunos meses después, creando un ambiente de incertidumbre acerca de los resultados. Estas son evaluaciones que constan  de  preguntas  tipo  opción  múltiple  las  cuales  requieren  de  conocimiento académicos previos para poder responder correctamente. Por ejemplo una evaluación con esas características son las evaluaciones que miden el nivel en un idioma extranjero  como  puede  ser  el  Inglés  (ECCE,MET,etc).  En  este  caso  los  resultados de estas evaluaciones son entregadas 2 meses después de haber sido aplicadas.
\vskip 0.2cm 
Esta investigación tiene como objetivo principal la predicción de la respuesta correcta de las preguntas de este tipo de evaluaciones. Para ello se usaran técnicas de aprendizaje no supervisado (Algoritmo K-means) asociado con la experiencia previa. Como experiencia previa se usará las calificaciones anteriores obtenidas de los 2 últimos exámenes de los estudiantes. Los resultados muestran que es posible estas predicciones mostrando un porcentaje de acierto de XX \%.

\vskip 0.3cm
\hspace*{-0.6cm}{\bf Palabras claves:} predicción, clusters, respuesta correcta, aprendizaje no supervisado.
\end{quotation}
%%%%%%%%%%%%%%%%%%%%%%%%%%%%%%%%%%%%%%%%%%%%%%%%%%%%%%%%%%%%%%%%%%%%%%%%%%%%%%%%%%%%


%%%%%%%%%%%%%%%%%%%%%%%%%%%%ABSTRACT%%%%%%%%%%%%%%%%%%%%%%
\newpage
\begin{center}
 \addcontentsline{toc}{chapter}{Abstract}
 {\bf\LARGE Abstract}\vskip 1.5cm
\end{center} 
\begin{quotation}

Nowadays, There are some tests which are taken in a specific date and their results are showed some months later. It creates a state of uncertainty about those results. These are tests which have multiple choice questions and previous knowledge is required to solve them. For example, a test with these characteristic is the test to measure the English level such as MET, ECCE and others. For these exams their results are showed two months later. 
\vskip 0.2cm
The main goal of this research is the prediction of the right answer of a multiplechoice question. To get this objective unsupervised learning algorithms (K-means)linked with previous experience will be used. In this case the previous experience will be the two last student’s grades.
The outcomes of this research show XX \% right so these predictions are correct.
\vskip 0.3cm
\hspace*{-0.6cm}{\bf Keywords:} prediction, cluster, rigth answer, unsupervised learning.
\end{quotation}
%%%%%%%%%%%%%%%%%%%%%%%%%%%%%%%%%%%%%%%%%%%%%%%%%%%%%%%%%%%%%%%%%%%%%%%%%%%%%%


%%%%%%%%%%%%%%%%%%%%%%%%%%% LISTA DE SIMBOLOS %%%%%%%%%%%%%%%%%%%%%%
\newpage
\addcontentsline{toc}{chapter}{Lista de símbolos}
 {\bf\LARGE Lista de símbolos}
 \vskip 1.5cm
Constantes: 
\begin{enumerate}
\item[(1)]$r,\overline{r} $ \hspace*{0.8cm} Indice que denota regiones.
\item[(2)] $n $ \hspace*{1.1cm} Indice de bienes finales deseados por los consumidores.
\item[(3)] ...
\vskip 3cm
\end{enumerate} 
\vskip 0.3cm
Variables:
\begin{enumerate}
\item[(5)] $ x^{r} $ \hspace*{1cm} Vector columna que denota la actividad de producción.
\item[(6)] $ u^{r} $ \hspace*{1.2cm} . . .
\end{enumerate}

